\documentclass[polish, 11pt]{article}
    
\usepackage[a4paper, margin=25mm]{geometry}
\usepackage{babel,polski}
\usepackage[utf8]{inputenc}
\usepackage[T1]{fontenc}
\usepackage{booktabs,multirow,multicol}

\usepackage{float}
\usepackage{graphicx}
\graphicspath{}

\usepackage{xcolor}
\usepackage[font=small,labelfont=bf]{caption}
\captionsetup[figure]{name=Rys.}

\newcommand{\results}[3][1.0]{
	\includegraphics[width=#1\textwidth]{#2}
	\captionof{figure}{#3\label{fig:#2}}
}
\setcounter{tocdepth}{4}
\setcounter{secnumdepth}{4}
\renewcommand{\thesubsubsection}{\alph{subsubsection})}
\renewcommand{\theparagraph}{\textbullet}       % kropki czy kolejne rzymskie czy coś innego?
%\renewcommand{\theparagraph}{\Roman{paragraph}}

\begin{document}
\begin{titlepage}
    \centering
    {\scshape\LARGE Bazy Danych\\ projekt \par}
    \vspace{1cm}
   
    {\itshape\Large Janusz Długosz --- 235746\/\par}
    {\itshape\Large Jakub Dorda --- 235013\/\par}
    {\itshape\Large Marcin Kotas --- 235098\/\par}
    \vfill
    Prowadzący:\par
    Dr inż.~Dariusz \textsc{Jankowski}

    \vfill

    {\large Wrocław, \today\par}

\end{titlepage}

\tableofcontents
\newpage

\section{Wstęp teoretyczny}
    \subsection{Podstawy relacyjnych baz danych}

    \subsection{Normalizacja}

\section{Część praktyczna projektu}
    \subsection{Przedstawienie problemu}

    \subsection{Wymagania systemu}
    
    \subsection{Model danych ERD}
        \subsubsection{Identyfikacja zbioru encji wraz z ich atrybutami kluczowymi}
            W systemie wyodrębnione zostały następujące encje oraz identyfikujące je atrybuty:
            \begin{table}[h]
                \centering
                \caption{Encje i identyfikatory}\label{tab:entitiesID}
                \begin{tabular}{cc}\toprule
                    Encja   	    &	Atrybut	\\\midrule
                    Locations	    &	Location\_ID	\\
                    Agencies	    &	Agency\_ID	\\
                    Users   	    &	User\_ID	\\
                    Apartments	    &	Apartment\_ID	\\
                    Reservations	&	Reservation\_ID	\\
                    Payments	    &	Payment\_ID	\\
                \bottomrule
                \end{tabular}
            \end{table}

        \subsubsection{Identyfikacja bezpośrednich zależności między encjami}
            Badane jest bezpośrednie powiązanie pomiędzy wszystkimi parami obiektów w systemie.
            Biorąc pod uwagę każdą parę obiektów A i B sprawdzamy czy każde wystąpienie obiektu typu A
            może być powiązane z pewnym wystąpieniem obiektu typu B bez udziału innego pośredniego obiektu.
            Uwzględniamy jednak tylko powiązania istotne z punktu widzenia systemu,
            to znaczy takie powiązania, o których informacja powinna być utrzymana w systemie,
            ponieważ odzwierciedlają one reguły i zasady panujące w organizacji użytkownika.
            Wykryte powiązania przedstawiliśmy w tak zwanej tablicy krzyżowej (tablica~\ref{tab:cross}),
            w których znakiem X oznaczono istnienie bezpośredniego, istotnego powiązania.
            \begin{table}[h]
                \centering
                \caption{Tabela krzyżowa, zależności bezpośrednie pomiędzy encjami}\label{tab:cross}
                \begin{tabular}{ccccccc}\toprule
                    &	Locations	&	Agencies	&	Users	&	Apartments	&	Reservations	&	Payments	\\\midrule
                    Locations	    &		&	X	&	X	&	X	&		&		\\
                    Agencies	    &	X	&		&		&	X	&		&		\\
                    Users	        &	X	&		&		&	X	&	X	&		\\
                    Apartments	    &	X	&	X	&	X	&		&	X	&		\\
                    Reservations	&		&		&	X	&	X	&		&	X	\\
                    Payments	    &		&		&		&		&	X	&		\\
                \bottomrule
                \end{tabular}
            \end{table}

            \begin{table}[H]
                \centering
                \caption{Opis atrybutów}\label{tab:attributes}
                \begin{tabular}{ll}\toprule
                    \multicolumn{1}{c}{Atrybut}	&	\multicolumn{1}{c}{Opis}\\\midrule
                    Location\_ID	&	Identyfikator lokalizacji	\\
                    Country\_Name	&	Nazwa państwa	\\
                    City	&	Nazwa miasta	\\
                    Longitude	&	Długość geograficzna	\\
                    Latitude	&	Szerokość geograficzna	\\
                    Street	&	Adres	\\
                    Street2	&	Adres c.d.	\\
                    Apartment\_Number	&	Nr mieszkania	\\
                    Zip\_Code	&	Kod pocztowy	\\
                    Agency\_ID	&	Identyfikator agencji	\\
                    Agencies.Name	&	Nazwa agencji	\\
                    Agencies.Info	&	Informacje o agencji	\\
                    Phone	&	Nr telefonu	\\
                    Web	&	Strona internetowa	\\
                    User\_ID	&	Identyfikator użytkownika	\\
                    Users.Type	&	Typ użytkownika (właściciel | klient)	\\
                    Email	&	Adres email	\\
                    Password	&	Hasło użytkownika	\\
                    Users.Info	&	Informacje o użytkowniku	\\
                    Users.Name	&	Imię użytkownika	\\
                    Last\_Name	&	Nazwisko użytkownika	\\
                    Apartment\_ID	&	Identyfikator apartamentu	\\
                    Cost\_Per\_Night	&	Koszt za jedną noc	\\
                    Bed\_Count	&	Liczba łóżek w apartamencie	\\
                    Air\_Cond	&	Klimatyzacja w apartamencie na wyposażeniu	\\
                    Standard	&	Standard apartamentu (1 --- 5)	\\
                    Owner\_ID	&	Identyfikator właściciela apartamentu	\\
                    Reservation\_ID	&	Identyfikator rezerwacji	\\
                    Date\_Begin	&	Data rozpoczęcia pobytu	\\
                    Date\_End	&	Data zakończenia pobytu	\\
                    Reservations.Status	&	Status rezerwacji (rozpoczęta | zaakceptowana | zakończona)	\\
                    Payment\_ID	&	Identyfikator wpłaty	\\
                    Payments.Type	&	Typ wpłaty (zaliczka | całość)	\\
                    Value	&	Wartość wpłaty	\\
                    Payments.Status	&	Status wpłaty (rozpoczęta | zakończona | anulowana | zwrócona)	\\
                \bottomrule
                \end{tabular}
            \end{table}

            \begin{table}[H]
                \centering
                \caption{Wykaz encji i powiązanych z nimi atrybutów}\label{tab:entitiesAttributes}
                \begin{tabular}{lp{0.8\linewidth}}\toprule
                    \multicolumn{1}{c}{Encja}	&	\multicolumn{1}{c}{Atrybut}	\\\midrule
                    Locations	&	Location\_ID, Country\_Name, City, Longitude, Latitude, Street, Street2, Apartment\_Number, Zip\_Code	\\
                    Agencies	&	Agency\_ID, Location\_ID, Name, Info, Phone, Web	\\
                    Users	&	User\_ID, Type, Email, Password, Info, Phone, Name, Last\_Name, Location\_ID	\\
                    Apartments	&	Apartment\_ID, Cost\_Per\_Night, Bed\_Count, Location\_ID, Agency\_ID, Air\_Cond, Standard, Owner\_ID	\\
                    Reservations	&	Reservation\_ID, Date\_Begin, Date\_End, Apartment\_ID, User\_ID, Status	\\
                    Payments	&	Payment\_ID, Reservation\_ID, Type, Value, Status	\\
                \bottomrule
                \end{tabular}
            \end{table}

            \begin{table}[H]
                \centering
                \caption{Opis związków pomiędzy encjami}\label{tab:relations}
                \begin{tabular}{ll}\toprule
                    \multicolumn{1}{c}{Związek}	&	\multicolumn{1}{c}{Opis}	\\\midrule
                    Posiada	&	Agencies --- Locations; 1:1	\\
                    Posiada	&	Users --- Locations; 1:1	\\
                    Posiada	&	Apartments --- Locations; 1:1	\\
                    Ma	&	Apartments --- Agencies; N:1	\\
                    Ma	&	Apartments --- Users; N:1	\\
                    Dotyczy	&	Reservations --- Apartments; N:1	\\
                    Dotyczy	&	Reservations --- Users; N:1	\\
                    Dotyczy	&	Payments --- Reservations; 1:1	\\
                \bottomrule
                \end{tabular}
            \end{table}

    \subsection{Schemat diagramu ERD}
	    \subsubsection{Opis aplikacji w której modelowano schemat}
	    
	    \subsubsection{Prezentacja schematu ERD bazy danych}

    \subsection{Rozwiązanie problemu}
	    \subsubsection{system bazodanowy}
		    \paragraph{Utworzenie bazy danych}
		    
		    \paragraph{Implementacja obiektów}
		    
		    \paragraph{Wprowadzenie danych}
		    
		    \paragraph{Zdefiniowanie typowych operacji SQL}

    \subsection{Podsumowanie}
	    \subsubsection{Ocena realizacji tematu}
	    
		\subsubsection{Wnioski}

\section{Literatura}

\end{document}
\documentclass[polish, 11pt]{article}
    
\usepackage[a4paper, margin=25mm]{geometry}
\usepackage{babel,polski}
\usepackage[utf8]{inputenc}
\usepackage[T1]{fontenc}
\usepackage{booktabs,multirow,multicol}

\usepackage{graphicx}
\graphicspath{}

\usepackage{xcolor}
\usepackage[font=small,labelfont=bf]{caption}
\captionsetup[figure]{name=Rys.}

\newcommand{\results}[3][1.0]{
	\includegraphics[width=#1\textwidth]{#2}
	\captionof{figure}{#3\label{fig:#2}}
}

\begin{document}
\begin{titlepage}
    \centering
    {\scshape\LARGE Bazy Danych\\ projekt \par}
    \vspace{1cm}
   
    {\itshape\Large Janusz Długosz --- 235746\/\par}
    {\itshape\Large Jakub Dorda --- 235013\/\par}
    {\itshape\Large Marcin Kotas --- 235098\/\par}
    \vfill
    Prowadzący:\par
    ~Dr inż.~Dariusz \textsc{Jankowski}

    \vfill

    {\large Wrocław, \today\par}

\end{titlepage}

\tableofcontents
\newpage

\section{Wstęp teoretyczny}
    \subsection{Podstawy relacyjnych baz danych}

    \subsection{Normalizacja}

\section{Część praktyczna projektu}
    \subsection{Przedstawienie problemu}

    \subsection{Wymagania systemu}
    
    \subsection{Model danych ERD}
		\subsubsection{Identyfikacja zbioru encji wraz z ich atrybutami kluczowymi}
	    
	    \subsubsection{Identyfikacja bezpośrednich zależności między encjami}
    
    \subsection{Schemat diagramu ERD}
	    \subsubsection{Opis aplikacji w której modelowano schemat}
	    
	    \subsubsection{Prezentacja schematu ERD bazy danych}

    \subsection{Rozwiązanie problemu}
	    \subsubsection{system bazodanowy}
		    \subparagraph{Utworzenie bazy danych}
		    
		    \subparagraph{Implementacja obiektów}
		    
		    \subparagraph{Wprowadzenie danych}
		    
		    \subparagraph{Zdefiniowanie typowych operacji SQL}

    \subsection{Podsumowanie}
	    \subsubsection{Ocena realizacji tematu}
	    
		\subsubsection{Wnioski}

\section{Literatura}

\end{document}
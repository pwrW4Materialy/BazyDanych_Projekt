\documentclass[polish, 11pt]{article}
    
\usepackage[a4paper, margin=25mm]{geometry}
\usepackage{babel,polski}
\usepackage[utf8]{inputenc}
\usepackage[T1]{fontenc}
\usepackage{booktabs,multirow,multicol}

\usepackage{graphicx}
\graphicspath{}

\usepackage{xcolor}
\usepackage[font=small,labelfont=bf]{caption}
\captionsetup[figure]{name=Rys.}

\newcommand{\results}[3][1.0]{
	\includegraphics[width=#1\textwidth]{#2}
	\captionof{figure}{#3\label{fig:#2}}
}

\begin{document}
\begin{titlepage}
    \centering
    {\scshape\LARGE Bazy Danych\\ projekt \par}
    \vspace{1cm}
   
    {\itshape\Large Janusz Długosz --- 235746\/\par}
    {\itshape\Large Jakub Dorda --- 235013\/\par}
    {\itshape\Large Marcin Kotas --- 235098\/\par}
    \vfill
    Prowadzący:\par
    ~Dr inż.~Dariusz \textsc{Jankowski}

    \vfill

    {\large Wrocław, \today\par}

\end{titlepage}

\tableofcontents
\newpage

\section{Wstęp teoretyczny}
    \subsection{Podstawy relacyjnych baz danych}
    Relacyjna baza danych jest zbiorem schematów tabel i relacji między nimi. Służą to przechowywania przeważnie dużych 
    zbiorów danych w ściśle określony sposób.
    Relacyjne bazy danych są jednym z trzech istniejących komercyjnie modelów, oprócz niej istnieją również bazy hierarchiczne
    i obiektowe, jednak nie cieszą się one tak dużą popularnością. Model relacyjny został opracowany przez Edgara Frank Codda
    w latach 70 ubiegłego wieku.
    Model organizuje dane w jedną lub więcej tabel składającą się z kolumn i rekordów (wierszy). Każdy rekord musi posiadać
    unikalny klucz, który go identyfikuje. Generalnie jedna tabela reprezentuje jeden typ encji (np. budynek).
    Rekordy przedstawiają instancje encji (np. ratusz), a w kolumnach przechowywane są atrybuty i informacje (np. rozmiar, rok budowy).
    	\subparagraph{Tabela\\}
    	To struktura przechowująca dane ściśle określonego typu. Tabela zawiera rekordy, które posiadają swoje atrybuty.
    	Struktury te można łączyć relacjami.
	  	\subparagraph{Klucze\\}
  		Każdy rekord posiada swój unikalny klucz. Pozwalają one na jednoznaczną identyfikację wiersza.
  	 Dzięki tym kluczom można łączyć rekordy między różnymi tabelami. To umożliwia modelowanie relacji. 
  		\begin{itemize}
  	 	\item \textbf{KLUCZ PODSTAWOWY (PRIMARY KEY)}\\
  	 	 To klucz identyfikujący jednoznacznie wiersz. Tabela może mieć tylko jeden taki klucz (nie może się powtarzać).
  	 	 Klucze te dzielą się na naturalne oraz sztuczne. Naturalne takie jak np. numer PESEL lub e-mail(koniecznie musi być unikalny) 
  	 	 jest z punktu widzenia systemu tak samo jak inne atrybuty jak np. nazwa firmy. Jeśli nie istnieje rzeczywisty identyfikator 
  	 	 to nadawany jest klucz sztuczny, który znajduję się w dodatkowo stworzonej kolumnie z możliwie jak najkrótszym kluczem.
  	 	 Zazwyczaj będą to kolejne liczby naturalne. 
  	 	 \item \textbf{KLUCZ OBCY (FOREIGN KEY)}\\
  	 	 Są to atrybuty, które wskazują na klucz podstawowy w innej tabeli, jest to po prostu relacja między dwoma tabelami.
  	 	 W tabeli, która jest powiązana kluczem obcym należy powielić całą strukturę, aby możliwe było wiązanie rekordów z
  	 	 dwóch różnych tabel. Z definicji pilnowane jest, aby w wartościach klucza obcego, mogły się znaleźć tylko wartości
  	 	 rzeczywiście istniejące jako klucz główny w innej tabeli. Klucz obcy może oczywiście dotyczyć również tej samej tabeli.
  	 	\end{itemize}
  	 	
  	 	\subparagraph{Relacje\\}
  	 	Relacje opisują związki między tabelami. Dobrze zaprojektowane relacje znacznie upraszczają prawidłowe działanie bazy danych,
  	 	poziom skomplikowania i czytelność kwerend. Poprzez nie  projektowana jest odpowiednia logika struktury bazy.
  	 	\begin{itemize}
  	 	\item \textbf{RELACJA 1:1}\\
  	 	Każdy rekord w pierwszej tabeli może mieć tylko jednego odpowiednika w drugiej tabeli i vice versa. 
  	 	Taką sytuację można rozpatrywać jako jedną dużą tabelę podzieloną na dwie mniejsze. Stosowany gdy 
  	 	część atrybutów tabeli można oddzielić, ponieważ może zostać użyta jako część innej tabli lub dotyczy 
  	 	tylko części atrybutów całej tabeli. Działanie takie poprawia czytelność oraz funkcjonalność bazy danych. Można 
  	 	też wydzielić część atrybutów, które są rzadko odpytywane.
  	 	\item \textbf{RELACJA 1:N}\\
  	 	Jest to relacja występująca najczęściej w relacyjnym modelu bazy danych. Występuje kiedy jeden element pierwszego 
  	 	zbioru może zostać powiązany z wieloma elementami zbioru drugiego. 
  	 	\item \textbf{RELACJA N:N}\\
  	 	Realizowany jest jako dwie relacje 1:N. Jeśli między tabelą pierwszą, a drugą ma zostać zaprojektowana taka relacja to potrzebna jest jeszcze 
  	 	trzecia tabela, która będzie pełniła funkcję łącznika. 
  	 	\end{itemize}
  	 	\subparagraph{Operacje na tabelach\\}
  	 	Są to operacje, które można wykonać w relacyjnych bazach danych. Pierwsze cztery bazują na matematycznej teorii mnogości.
  	 	\begin{itemize}
  	 	\item \texttt{UNION} - suma zbiorów, zwraca wszystkie rekordy pierwszej i drugiej tabeli bez duplikatów.
  	 	\item \texttt{INTERSECTION} - iloczyn zbiorów, zwraca tylko rekordy będące częścią wspólną pierwszej i drugiej tabeli.
  	 	\item \texttt{EXCEPT} - różnica zbiorów, zwraca rekordy z pierwszej tabeli bez części wspólnej pierwszej i drugiej tabeli.
  	 	\item \texttt{CROSS JOIN} - iloczyn kartezjański, zwraca iloczyn kartezjański dwóch tabel, czyli wszystkie możliwe kombinacje połączenia
  	 	rekordów tych tabel.
  	 	\item \texttt{SELECT} + \texttt{WHERE} - zwraca wybrane rekordy z tabeli, które spełniają określony warunek.
  	 	\item \textbf{PROJECTION OPERATION} - zwraca tylko wybrane atrybuty z rekordów.
  	 	\item \texttt{JOIN} - zwraca połączone tabele, które łączy relacja.
  	 	\item \textbf{DIVISION} - jest operacją przeciwną do ilorazu kartezjańskiego.
  	 	\end{itemize}
  		
  		
    

    \subsection{Normalizacja}

\section{Część praktyczna projektu}
    \subsection{Przedstawienie problemu}

    \subsection{Wymagania systemu}
    
    \subsection{Model danych ERD}
		\subsubsection{Identyfikacja zbioru encji wraz z ich atrybutami kluczowymi}
	    
	    \subsubsection{Identyfikacja bezpośrednich zależności między encjami}
    
    \subsection{Schemat diagramu ERD}
	    \subsubsection{Opis aplikacji w której modelowano schemat}
	    
	    \subsubsection{Prezentacja schematu ERD bazy danych}

    \subsection{Rozwiązanie problemu}
	    \subsubsection{system bazodanowy}
		    \subparagraph{Utworzenie bazy danych}
		    
		    \subparagraph{Implementacja obiektów}
		    
		    \subparagraph{Wprowadzenie danych}
		    
		    \subparagraph{Zdefiniowanie typowych operacji SQL}

    \subsection{Podsumowanie}
	    \subsubsection{Ocena realizacji tematu}
	    
		\subsubsection{Wnioski}

\section{Literatura}

\end{document}